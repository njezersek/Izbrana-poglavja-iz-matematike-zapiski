\documentclass[11pt,a4paper]{article}
\usepackage[utf8x]{inputenc}

\usepackage{fancyhdr}

\usepackage[pdftex]{graphicx} % Required for including pictures
\usepackage[pdftex,linkcolor=black,pdfborder={0 0 0}]{hyperref} % Format links for pdf
\usepackage{calc} % To reset the counter in the document after title page
\usepackage{enumitem} % Includes lists

\usepackage{textcomp}
\usepackage{eurosym}

\usepackage{graphicx}
\graphicspath{ {./images/} }

\usepackage{ amssymb } % extra math symbols
\usepackage{ amsmath } % extra math symbols
\usepackage{amsthm}
\usepackage{ dsfont } % font za množice
% tabele
\usepackage{array}
\usepackage{wrapfig}
\usepackage{multirow}
\usepackage{tabularx}
\usepackage{multicol}
\usepackage{listings}
\usepackage{caption}
\usepackage{tikz,forest}
\usetikzlibrary{arrows.meta}

\newtheorem*{trditev}{Trditev}
\newtheorem*{izrek}{Izrek}
\newtheorem*{posledica}{Posledica}
\newtheorem*{definicija}{Definicija}

\frenchspacing % No double spacing between sentences
\setlength{\parindent}{0pt}
\setlength{\parskip}{0.5em}

\usepackage{mathtools}
\usepackage{blkarray, bigstrut} %

%\pagenumbering{gobble}

\usepackage[a4paper, lmargin=0.1666\paperwidth, rmargin=0.1666\paperwidth, tmargin=0.1111\paperheight, bmargin=0.1111\paperheight]{geometry} %margins

\DeclareCaptionLabelFormat{algocaption}{Algorithm \thenalg} % defines a new caption label as Algorithm x.y

\lstnewenvironment{algorithm}[1][] %defines the algorithm listing environment
{   
    %\refstepcounter{nalg} %increments algorithm number
    %\captionsetup{labelformat=algocaption,labelsep=colon} %defines the caption setup for: it ises label format as the declared caption label above and makes label and caption text to be separated by a ':'
    \lstset{ %this is the stype
        mathescape=true,
        %frame=tB,
        %numbers=left, 
        numberstyle=\tiny,
        basicstyle=\scriptsize, 
        keywordstyle=\color{black}\bfseries\em,
        keywords={,vhod, izhod, vrni, dokler, izvajaj, konec, zanke} %add the keywords you want, or load a language as Rubens explains in his comment above.
        %numbers=left,
        %xleftmargin=.04\textwidth,
        %#1 % this is to add specific settings to an usage of this environment (for instnce, the caption and referable label)
    }
}
{}
\begin{document}
	\section{Kolobarji polinomov}
	\subsection*{Računanje s kompleksnimi števili}
	\begin{gather*}
		z = x + iy = r e^{i\varphi} = r\left( \cos \varphi + i \sin \varphi \right)
	\end{gather*}
	\begin{align*}
		r = |z| = \sqrt{x^2 + y^2} && \varphi = \arg z = \arctan \frac{y}{x}
	\end{align*}

	\subsubsection*{De Moivreova formula}
	\begin{align*}
		z^n = r^n\left( \cos \varphi n + i \sin \varphi n \right)
	\end{align*}

	\subsubsection*{Osnovni izrek algebre}
	Vsak nekonstanten polinom $a_n x^n + \dots + a_0$ ima natanko $n$ kompleksnih ničel (štetih z večkratnostjo).

	\subsection*{Trigonometrične identitete}
	\begin{align*}
		&\sin(x \pm y) = \sin(x) \cos(y) \pm \cos(x) \sin(y) \\
		&\cos(x \pm y) = \cos(x) \cos(y) \mp \sin(x) \sin(y)\\
		&\tan(x \pm y) = \frac{\tan(x)\pm \tan(y)}{1 \mp \tan(x) \tan(y)}\\
		&\cot(x \pm y) = \frac{\cot(x)\cot(y) \mp 1}{\tan(x) \pm \tan(y)}\\
		&\sin^2(x)+\cos^2(x) = 1\\
		&1+\cot^2(x) = \frac{1}{\sin^2(x)}\\
		&1+\tan^2(x) = \frac{1}{\cos^2(x)}\\
		&\sin\frac{x}{2} = \pm\sqrt{\frac{1-\cos x}{2}}\\
		&\cos\frac{x}{2} = \pm\sqrt{\frac{1+\cos x}{2}}\\
	\end{align*}

	\subsection*{Mali Fermantov izrek}
	\begin{align}
		\forall a \in \mathbb{Z}, p \in \mathbb{P}:\ a^p \equiv_p a
	\end{align}

	\subsection*{Polinomi}
	Polinom je \textbf{razcepen}, če ga lahko zapišemo kot produkt dveh nekonstantnih polinomov.
	Nekonstanten polinom, ki ni razcepen je \textbf{nerazcepen}.

	Polinom $a_n x^n + \dots + a_0$ je \textbf{primitiven}, če velja $\gcd(a_0, \dots, a_n) = 0$

	\subsubsection*{Gaussova lema}
	\begin{align}
		p(x) \in \mathbb{Z}[x] \text{ razcepen nad } \mathbb{Z} \iff p(x) \text{ razcepen nad } \mathbb{Q}
	\end{align}

	\subsubsection*{Hornerjev algoritem}
	\[a_n x^n + ... + a_0 = 0\]
	\begin{itemize}
		\item možne cele ničle: $\pm$delitelji $a_0$
		\item možne racionalne ničle: $\pm \frac{\text{delitelji }a_0}{\text{delitelji }a_n} = k$
	\end{itemize}
	\begin{center}
		\begin{tabular}{ l|l l l l}
				& $a_n$ & $a_{n-1}$ & $...$ & $a_0$ \\ \hline
			$k$ &       & $ka_n$    & $...$ & \\ \hline
				& $a_n$ & $ka_n - a_{n-1}$ & $...$ & ostanek\\
		\end{tabular}
	\end{center}

	\subsubsection*{Eisensteinov kriterij}
	Naj bo $a(x) = a_n x^n + \dots + a_0 \in \mathbb{Z}[x]$ polinom. Če $\exists p \in \mathbb{P} : p | a_0, \dots, a_{n-1} \wedge p \nmid a_n \wedge p^2 \nmid a_0$, potem je $a(x)$ nerazcepen nad $\mathbb{Q}$.

	\subsubsection*{Rodovne funkcije}
	\[
		\begin{aligned}
			\sum_{n=0}^{\infty} q^n &= \frac{1}{1-q} &
			\sum_{n=0}^{b} q^n &= \frac{1-q^{b+1}}{1-q}
			\\
			\sum_{n=a}^{\infty} q^n &= \frac{q^{a}}{1-q} &
			\sum_{n=a}^{b} q^n &= \frac{q^a-q^{b+1}}{1-q}
		\end{aligned}
	\]
	\[
		a^n - b^n = (a-b)(a^{n-1} + a^{n-2}b + ... + ab^{n-2} + b^{n-1})  
	\]
	\[ \textstyle \frac{a_0 + ... + a_{k-1}x^{k-1}}{1-x^k} = a_0 + ... + a_{k-1}x^{k-1} + a_0^k + ... + a_{k-1}x^{2k-1} + ...\]
	\[ (x+y)^n = \sum_{k=0}^{n} \binom{n}{k} x^{n-k}y^{k} \]
	\[ \frac{1}{(1-x)^n} = \sum_{k=0}^{n} \binom{n+k-1}{k} x^{k} \]
	\[ B_\lambda(x) = \sum_{n} \binom{\lambda}{n} x^{n} = (1+x)^\lambda; \qquad \binom{\lambda}{n} = \frac{\lambda^{\underline{n}}}{n!}\]

\end{document}